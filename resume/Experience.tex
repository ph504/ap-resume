\cvsection{Experience}

\definecolor{cobalt}{rgb}{0.0, 0.28, 0.67}

\begin{cventries}
    \cventry
    {Human-Robot Interaction Laboratory \textemdash University of New Brunswick}
    {\textbf{RESEARCH ASSISTANT}}
    {UNB, Fredericton, Canada}
    {September 2023 - Present (2 years)}
    {
        % \vspace{0.3 cm}
        % Selected Links: \hspace{0.1 cm} \href{https://hcilab.github.io/} {\textcolor{cobalt}{[org]}} \hspace{0.1 cm} 
        % \href{https://github.com/ph504/Teleoperation-Interface} {\textcolor{cobalt}{[code]}} \hspace{0.1 cm}
        % \href{https://ph504.github.io/projects/projects-1/} {\textcolor{cobalt}{[page]}} \hspace{0.1 cm}
        \vspace{0.2 cm}
        \begin{itemize}
            \item Designed and implemented teleoperation interfaces and control systems using ROS + Python (Tkinter, MVC) to improve responsiveness and human-robot collaboration.
            \item Conducted hardware diagnostics and sensing calibration on Clearpath’s Jackal robot (MCU/User Power Board, VBAT continuity, signal integrity).
            \item Simulated and evaluated robotic task performance, analyzing latency, control precision, and sensor feedback quality.
            \item Developed data pipelines for sensor fusion and real-time performance monitoring.
            \item Collaborated on empathetic robot design and autonomous behavior research to enhance operator safety and trust.
            \item visit \texttt{https://hcilab.github.io/} for more info.
        \end{itemize}
    }
\end{cventries}
%     \item Built teleoperation interfaces with ROS + Python (Tkinter); implemented an MVC architecture to improve reliability and developer velocity.
%     % \item Investigated connectivity and networking issues (routing, DNS, ROS master) and implemented automated recovery routines to stabilize robot communication.
%     % % \item Performed hardware diagnostics on the Jackal’s MCU/User Power Board—re-terminating connectors, verifying VBAT/power continuity, and preventing intermittent disconnections.
%     % \item Maintained and refactored a legacy codebase, improving readability, modularity, and documentation quality.
%     % \item Contributed to ongoing research on UI/UX design, system robustness, and human-robot interaction for teleoperation studies. (Empathy with Teleoperated Robots, and Mitigating Racial and Gender Bias Using Avatar Robots)
%     % \item Worked with Python TKinter, ROS, ShellScripts
% \begin{itemize}
%     \item Built teleoperation interfaces
% \end{itemize}
% \vspace{0.4 cm}
    

        
        % \begin{cventries}
        %     \cventry
        %     {\textbf{RESEARCH ASSOCIATE}}
        %     {Human-Robot Interaction Laboratory}
        %     {UNB, Fredericton, Canada}
        %     {May - September 2024}
        %     {Technical Lead, co-author, designer, and conductor of the studies of underlying unconscious biases through interactions via avatars and teleoperated robots, in a professional setting.
        %     }
        %     \vspace{0.4 cm}
        % \end{cventries}
        
        % \begin{cventries}
        %     \cventry
        %     {Robotics Laboratory \textemdash Shahid Beheshti University}
        %     {\textbf{RESEARCH ASSISTANT}}
        %     {SBU, Tehran, Iran}
        %     {July 2021 - October 2023 (2 years 3 months)}
        %     {
            %         \begin{itemize}
            %             \item Explored and evaluated D3QN, PPO, and genetic algorithms for: 
            %             \begin{enumerate}[label=(\alph*)]    
            %                 \item improving value-function estimates of RCSS2D intelligent agents, and
            %                 \item robotic-arm path planning (see Publications).
            %             \end{enumerate}
            %             \item Integrating C++ RCSS system with PyTorch and Python modules, using Redis-based local networking and sensor data exchange.
            %             % \item  Worked with C++, Pandas, TensorFlow, PyTorch, ROS, Webots, and Docker
            %         \end{itemize}
            %     }
            %     \vspace{0.4 cm}
            % \end{cventries}
            
            
\begin{cventries}
    \cventry
    {Robotics Laboratory \textemdash R3SBU Team}
    {\textbf{TEAM LEAD}}
    {SBU, Tehran, Iran}
    {September 2021 - August 2023 (2 years)}
    {
        % Selected Links: \hspace{0.1 cm} [org] : github.com/cserobotic \hspace{0.1 cm} also visit the github page.
        % Selected Links: \hspace{0.1 cm} \href{https://github.com/cserobotic}{\textcolor{cobalt}{[org]}} \hspace{0.1 cm}
        % \href{https://ph504.github.io/projects/projects-3/}{\textcolor{cobalt}{[page]}} \hspace{0.1 cm}
        \begin{itemize}
            \item Led a 9-member interdisciplinary robotics team developing autonomous and semi-autonomous robots (RoboCup 2D, quadcopter, SBU omni-robot, and Humanoid NAO).
            \item Designed simulation and testing frameworks for robot motion and vision system evaluation.
            \item Integrated sensor data, motion planning, and control algorithms across C++ and Python environments.
            \item Managed Git workflows, documentation, and milestone delivery across multiple concurrent projects.
            \item Built and maintained a semi-automated CI/CD-style validation pipeline, performing statistical analyses on sparse datasets to generate actionable QA reports.
            \item Secured sponsorship from Divar grant to support lab research and competitions.
            \item Visit \texttt{https://github.com/cserobotic} for more info.
        \end{itemize}
        }
\end{cventries}
            % \item Team repos — integration, CI, and analysis; core repo is private (under NDA).
            % \href{https://github.com/cserobotic}                        {\textcolor{cobalt}{[org]}} \hspace{0.1 cm}
            % \href{https://github.com/cserobotic/R3-autotest}            {\textcolor{cobalt}{[pipeline]}} \hspace{0.1 cm}
            % \href{https://github.com/cserobotic/advanced-log-analyzer}  {\textcolor{cobalt}{[RL experiments]}} \hspace{0.1 cm}
        % \href{https://github.com/cserobotic/R3-log-analyzer}        {\textcolor{cobalt}{[Data Processing]}} \hspace{0.1 cm}
                
\begin{cventries}
    \cventry
    {Robotics Laboratory \textemdash R3SBU Team}
    {\textbf{ROBOTICS ENGINEER INTERN}}
    {SBU, Tehran, Iran}
    {May - September 2021 (4 months)}
    {
        % Selected Links: \hspace{0.1 cm} \href{https://github.com/ph504/usb_omni_bot} {\textcolor{cobalt}{[code]}} \hspace{0.1 cm}
        % \href{https://ph504.github.io/projects/projects-2/} {\textcolor{cobalt}{[page]}} \hspace{0.1 cm}
        \begin{itemize}
            \item Developed and tested autonomous navigation algorithms (BUG1/BUG2, SLAM) using Webots and ROS.
            \item Improved sonar-based obstacle detection (corner-miss issue) via semi-circular sweep motion; compared against sensor-swap and decoupled-panel alternatives for accuracy and energy.
            \item Modeled robotic systems in simulation for collision detection and motion validation.
            \item Set up ROS 2 platform and packages; worked with networking, nodes, and launch configs.
            \item Fixed Webots 3D model issues (URDF/PROTO) on the SBU omni-directional robot to resolve dynamic-motion failures.
            \item Visit \texttt{https://ph504.github.io/projects/projects-4/} for more info.
        \end{itemize}
    }
\end{cventries}
                              
\begin{cventries}
    \cventry
    {Concealand Game Studio}
    {\textbf{GAME DEVELOPER INTERN}}
    {Tehran, Iran}
    {January - May 2023 (6 months)}
    {
        \begin{itemize}    
            \item Applied Reinforcement Learning for Procedural Animation of a humanoid character to significantly reduce the animation state complexity and workload for artists.
            \item Tuned PyTorch training pipelines to reduce GPU usage and improve convergence time.
            \item Worked with Unity IK frameworks, animation rigging package, and Unity ML model training
        \end{itemize}
    }
\end{cventries}

% \vspace{0.4 cm}
% \cvsection{Teaching}
% \begin{cventries}
%     \cventry
%     {Introduction to Algorithms Design}
%     {\textbf{TEACHING ASSISTANT}}
%     {Shahid Beheshti University, Tehran, Iran}
%     {September 2022 - December 2022}
%     {
    %         \begin{itemize}
    %             \item Instructor: Ramak Ghavamizadeh; Responsible for teaching labs and designing lab assignment problemsets.
    %             \item Used C++ and bash scripts to automate the test units for marking while assessing edge cases in each problem.
    %         \end{itemize}
    %     }
    %     \vspace{0.4 cm}
    % \end{cventries}
    
    % \begin{cventries}
    %     \cventry
    %     {Digital Logic Circuits}
    %     {\textbf{TEACHING ASSISTANT}}
    %     {Shahid Beheshti University, Tehran, Iran}
    %     {September 2022 - December 2022}
    %     {
        %         \begin{itemize}
        %             \item Instructor: Hamidreza Mahdiani; Responsible for marking labs and assignments.
        %         \end{itemize}
        %     }
        %     \vspace{0.4 cm}
        % \end{cventries}
% \begin{cventries}
%     \cventry
%     {Introduction to Robotics}
%     {\textbf{INSTRUCTOR}}
%     {SBU, Tehran, Iran}
%     {Summers 2022 \& 2023}
%     {
%         \begin{itemize}
%             \item Instructed summer workshops to recruit and train new lab members.
%             \item Taught Fundamentals of Robotics, Machine Learning, AI Algorithms, and Simulation concepts, using C++ and Python
%         \end{itemize}
%     }
%     \vspace{0.4 cm}
% \end{cventries}
                    
% \begin{cventries}
%     \cventry
%     {Algorithms Design and Analysis}
%     {\textbf{TEACHING ASSISTANT}}
%     {UNB, Fredericton, Canada}
%     {Winter 2025}
%     {
%         \begin{itemize}
%             \item Instructor: Huajie Zhang; Responsible for marking and reviewing assignments
%         \end{itemize}
%     }
%     \vspace{0.4 cm}
% \end{cventries}
            
% \begin{cventries}
%     \cventry
%     {Introduction to Game Development}
%     {\textbf{TEACHING ASSISTANT}}
%     {UNB, Fredericton, Canada}
%     {Winter 2024}
%     {
%         \begin{itemize}
%             \item Instructor: Daniel Rea; Responsible for marking and reviewing assignments, as well as tutorial outline with Godot Engine.
%         \end{itemize}
%     }
%     \vspace{0.4 cm}
% \end{cventries}
    
% \begin{cventries}
%     \cventry
%     {Introduction to Algorithms Design}
%     {\textbf{TEACHING ASSISTANT}}
%     {Shahid Beheshti University, Tehran, Iran}
%     {Fall 2022}
%     {
%         \begin{itemize}
%             \item Instructor: Ramak Ghavamizadeh; Responsible for teaching labs and designing lab assignment problemsets.
%             \item Used C++ and bash scripts to automate the test units for marking while assessing edge cases in each problem.
%         \end{itemize}
%     }
%     \vspace{0.4 cm}
% \end{cventries}
                                        % \begin{cventries}
                                        %     \cventry
                                        %     {Introduction to Robotics}
                                        %     {\textbf{INSTRUCTOR}}
                                        %     {SBU, Tehran, Iran}
                                        %     {June - August 2023
                                        %      June - August 2022}
                                        %     {
                                            %         \begin{itemize}
                                            %             \item Instructed for summer workshops for undergraduate students and lab recruitment.
                                            %             \item Taught Fundamentals of Robotics, Machine Learning, A.I. Algorithms, and Simulation concepts, Using Python and C++
                                            %         \end{itemize}
                                            %     }
                                            %     \vspace{0.4 cm}
                                            % \end{cventries}