\cvsection{Experience}


\begin{cventries}
    \cventry
    {Human-Robot Interaction Laboratory \textemdash University of New Brunswick}
    {\textbf{RESEARCH ASSISTANT}}
    {UNB, Fredericton, Canada}
    {September 2023 - Present (2 years)}
    {
        \begin{itemize}
            \item Built teleoperation interfaces with ROS + Python (Tkinter); implemented an MVC architecture to improve reliability and developer velocity.
            \item Investigated connectivity and networking issues (routing, DNS, ROS master) and implemented automated recovery routines to stabilize robot communication.
            % \item Performed hardware diagnostics on the Jackal’s MCU/User Power Board—re-terminating connectors, verifying VBAT/power continuity, and preventing intermittent disconnections.
            \item Maintained and refactored a legacy codebase, improving readability, modularity, and documentation quality.
            \item Contributed to ongoing research on UI/UX design, system robustness, and human-robot interaction for teleoperation studies. (Empathy with Teleoperated Robots, and Mitigating Racial and Gender Bias Using Avatar Robots)
            % \item Worked with Python TKinter, ROS, ShellScripts
        \end{itemize}}
        \vspace{0.4 cm}
\end{cventries}
    

        
        % \begin{cventries}
        %     \cventry
        %     {\textbf{RESEARCH ASSOCIATE}}
        %     {Human-Robot Interaction Laboratory}
        %     {UNB, Fredericton, Canada}
        %     {May - September 2024}
        %     {Technical Lead, co-author, designer, and conductor of the studies of underlying unconscious biases through interactions via avatars and teleoperated robots, in a professional setting.
        %     }
        %     \vspace{0.4 cm}
        % \end{cventries}
        
        % \begin{cventries}
        %     \cventry
        %     {Robotics Laboratory \textemdash Shahid Beheshti University}
        %     {\textbf{RESEARCH ASSISTANT}}
        %     {SBU, Tehran, Iran}
        %     {July 2021 - October 2023 (2 years 3 months)}
        %     {
            %         \begin{itemize}
            %             \item Explored and evaluated D3QN, PPO, and genetic algorithms for: 
            %             \begin{enumerate}[label=(\alph*)]    
            %                 \item improving value-function estimates of RCSS2D intelligent agents, and
            %                 \item robotic-arm path planning (see Publications).
            %             \end{enumerate}
            %             \item Integrating C++ RCSS system with PyTorch and Python modules, using Redis-based local networking and sensor data exchange.
            %             % \item  Worked with C++, Pandas, TensorFlow, PyTorch, ROS, Webots, and Docker
            %         \end{itemize}
            %     }
            %     \vspace{0.4 cm}
            % \end{cventries}
            
\begin{cventries}
    \cventry
    {Concealand Game Studio}
    {\textbf{GAME DEVELOPER INTERN}}
    {Tehran, Iran}
    {January - August 2023}
    {
        \begin{itemize}    
            \item Applied Reinforcement Learning for Procedural Animation of a humanoid character to significantly reduce the animation state complexity and workload for artists.
            \item Worked with Unity IK frameworks, animation rigging package, and ML model training
        \end{itemize}
    }
\end{cventries}

\begin{cventries}
    \cventry
    {Robotics Laboratory \textemdash Shahid Beheshti University}
    {\textbf{TEAM LEAD}}
    {SBU, Tehran, Iran}
    {September 2021 - August 2023 (2 years)}
    {
        \begin{itemize}
            \item Led a 9-member interdisciplinary team bridging hardware and AI through TCP-based networking, developing intelligent agents for both simulated and real robotic systems (RoboCup 2D, quadcopter, SBU omni-robot, and Humanoid NAO).
            \item Secured a sponsorship grant from Divar to support competition and lab activities.
            \item Managed Git workflows, code reviews, and technical design across multiple game-like simulation environments.
            \item Integrated and packaged modules for intelligent-system compatibility; ensured cohesive builds and cross-team integration.
            \item Built and maintained a semi-manual CI/CD-style validation pipeline, performing statistical analyses on sparse datasets to generate actionable QA reports.
        \end{itemize}
    }
\end{cventries}

% \begin{cventries}
%     \cventry
%     {Robotics Laboratory \textemdash Shahid Beheshti University}
%     {\textbf{ROBOTICS ENGINEER INTERN}}
%     {SBU, Tehran, Iran}
%     {May 2021 - September 2021 (5 months)}
%     {
%         \begin{itemize}
%             \item Implemented and evaluated bug algorithms and obstacle-avoidance (Webots, Python); later explored particle filters and SLAM for higher autonomy (Webots, ROS, Python).
%             \item Improved sonar-based obstacle detection (corner-miss issue) via semi-circular sweep motion; compared against sensor-swap and decoupled-panel alternatives for accuracy and energy.
%             \item Set up ROS 2 platform and packages; worked with networking, nodes, and launch configs.
%             \item Fixed Webots 3D model issues (URDF/PROTO) on the SBU omni-directional robot to resolve dynamic-motion failures.
%         \end{itemize}
%     }
% \end{cventries}



% \vspace{0.4 cm}
\cvsection{Teaching}
% \begin{cventries}
%     \cventry
%     {Introduction to Algorithms Design}
%     {\textbf{TEACHING ASSISTANT}}
%     {Shahid Beheshti University, Tehran, Iran}
%     {September 2022 - December 2022}
%     {
%         \begin{itemize}
%             \item Instructor: Ramak Ghavamizadeh; Responsible for teaching labs and designing lab assignment problemsets.
%             \item Used C++ and bash scripts to automate the test units for marking while assessing edge cases in each problem.
%         \end{itemize}
%     }
%     \vspace{0.4 cm}
% \end{cventries}

% \begin{cventries}
%     \cventry
%     {Digital Logic Circuits}
%     {\textbf{TEACHING ASSISTANT}}
%     {Shahid Beheshti University, Tehran, Iran}
%     {September 2022 - December 2022}
%     {
%         \begin{itemize}
%             \item Instructor: Hamidreza Mahdiani; Responsible for marking labs and assignments.
%         \end{itemize}
%     }
%     \vspace{0.4 cm}
% \end{cventries}
\begin{cventries}
    \cventry
    {Algorithms Design and Analysis}
    {\textbf{TEACHING ASSISTANT}}
    {UNB, Fredericton, Canada}
    {Winter 2025}
    {
        \begin{itemize}
            \item Instructor: Huajie Zhang; Responsible for marking and reviewing assignments
        \end{itemize}
    }
    \vspace{0.4 cm}
\end{cventries}
    
\begin{cventries}
    \cventry
    {Introduction to Game Development}
    {\textbf{TEACHING ASSISTANT}}
    {UNB, Fredericton, Canada}
    {Winter 2024}
    {
        \begin{itemize}
            \item Instructor: Daniel Rea; Responsible for marking and reviewing assignments, as well as tutorial outline
            \item Worked with Godot Engine.
        \end{itemize}
    }
    \vspace{0.4 cm}
\end{cventries}

\begin{cventries}
    \cventry
    {Introduction to Robotics}
    {\textbf{INSTRUCTOR}}
    {SBU, Tehran, Iran}
    {Summers 2022 \& 2023}
    {
        \begin{itemize}
            \item Instructed for summer workshops, in order to recruit new passionate individuals for the lab.
            \item Taught Fundamentals of Robotics, Machine Learning, A.I. Algorithms, and Simulation concepts, using C++ and Python
        \end{itemize}
    }
    \vspace{0.4 cm}
\end{cventries}

% \begin{cventries}
%     \cventry
%     {Introduction to Robotics}
%     {\textbf{INSTRUCTOR}}
%     {SBU, Tehran, Iran}
%     {June - August 2023
%      June - August 2022}
%     {
%         \begin{itemize}
%             \item Instructed for summer workshops for undergraduate students and lab recruitment.
%             \item Taught Fundamentals of Robotics, Machine Learning, A.I. Algorithms, and Simulation concepts, Using Python and C++
%         \end{itemize}
%     }
%     \vspace{0.4 cm}
% \end{cventries}